\documentclass[12pt]{article}  % [12pt] option for the benefit of aging markers
\usepackage{amssymb,amsthm}    % amssymb package contains more mathematical symbols
\usepackage{graphicx}          % graphicx package enables you to paste in graphics
\usepackage[capposition=top]{floatrow}
\usepackage{setspace}
\usepackage{float}
\usepackage[backend=bibtex,sorting=nyt,firstinits=true]{biblatex}
\usepackage{hyperref}
\usepackage{longtable}
\usepackage{wrapfig}
\usepackage[titletoc,title]{appendix}
\usepackage{pdfpages}






\usepackage{lipsum}
\usepackage[margin=2cm]{geometry}


\addbibresource{references.bib}
\newcommand{\rid}[1]{\centering #1-\ifnum\value{requirement}<10 0\fi\arabic{requirement} \stepcounter{requirement}}








%%%%%%%%%%%%%%%%%%%%%%%%%%%%%%%%%%%%%%%%%
%										%
%     			Title					%
%										%
%%%%%%%%%%%%%%%%%%%%%%%%%%%%%%%%%%%%%%%%%
\title{Digital Lab Marking System \\~\\  \large{Heriot-Watt University} \\~\\ Final Year Dissertation \\~\\ Masters of Software Engineering}
\author{Lewis McNeill\\
supervised by
Peter J King}






\begin{document}
\maketitle
\pagenumbering{gobble}
\newpage

\setcounter{page}{1}
\pagenumbering{roman}

\doublespacing
\textbf{\Large{Declaration}} \\[2em]
I, Lewis Francis McNeill, confirm that this work submitted for assessment is my own and is expressed in my own words. Any references, made within it, of the works of other authors in any way (e.g., ideas, equations, figures, text, tables, programs) are properly acknowledged at any point of their use. A list of the references employed is included.
\\
\\
Signed: Lewis McNeill
\\
Date: \today






\newpage                
\begin{abstract}


\noindent
The aim of this dissertation project is to replace the current system for the marking of computer labs with a new digital system. This will enable lecturers to create a marking scheme online. Lab helpers will select the student they are marking and the marking scheme will then be loaded, marks will be entered and then made immediately available to both student and lecturers to view. It will also provide useful statistics for both student and lecturers.


\end{abstract}


\newpage                 
\tableofcontents




%%%%%%%%%%%%%%%%%%%%%%%%%%%%%%%%%%%%%%%%%
%										%
%     		Introduction				%
%										%
%%%%%%%%%%%%%%%%%%%%%%%%%%%%%%%%%%%%%%%%%
\newpage    


\setcounter{page}{1}
\pagenumbering{arabic}


\section{Introduction}


The current system for marking of computing science lab is to use multiple lab helpers, each given a list of students and the marking scheme for them. Generally marking schemes are a selection of tasks students must have completed and lab helpers tick them off when they have completed them.. The biggest problem with this part of the marking is the time it takes lab helpers to locate the students on the list. This causes frustration with increased wait times for students.  Multiple other issues  can also  arise from this: students can be marked by two helpers and obtain different grade from both; the lab helper omits to  tick off a completed task; they assign  marks for the wrong student on the sheet or simply they misplace the actual marking sheet.


After the lab helpers have completed their marking, the sheets are provided to the lecturer who collates them all together into one spreadsheet to calculate marks. After that it is entered it into vision. This too can cause its own set of problems- it increases the chances of transcription errors as the lecturer may misread marks when they are transferring them across. The lecturer may not enter the marks immediately into the spreadsheet increasing the chance that a marking sheet goes missing, and finally this system means that students are having to wait even longer to receive their results.


The objective is to develop a system that will reduce and hopefully eliminate the problems of the current system. Along with this, it should attempt to reduce the amount of time taken to mark students’ work and therefore speed up labs in general. It should also enable students to see their grades immediately, allow lecturer to see the result of the assignments as they are being marked and make marking quicker for lab helpers.


\newpage
\section{Aims and Objectives}
\subsection{Aim}
The aim of this dissertation is to design and implement a system for the digital marking and analysis of computer labs and to help improve the speed at which they are marked. The system will also provide useful statistics for both lecturers and students.


\subsection{Objectives}
\label{section:object}
\begin{itemize}
\item Simplify the way that labs marks are currently processed.
\item Allow lecturers to create marking schemes online that lab helpers can access
\item Lab helpers can mark students in labs using marking schemes.
\item Develop a system that allows lab helpers to mark labs using an online application.
\item Allow students to see the mark they achieved from the lab instantly.
\item Provide useful statistics and graphs for lecturers.
\end{itemize}












%%%%%%%%%%%%%%%%%%%%%%%%%%%%%%%%%%%%%%%%%
.%										%
%     	   Literature Review			%
%		c								%
%%%%%%%%%%%%%%%%%%%%%%%%%%%%%%%%%%%%%%%%%




\newpage
\section{Literature Review}
This section contains the summaries of literature relating to the topic and should help to create a context for the development of a digital marking system . It will cover what marking systems are currently being used, what current digital marking systems  actually exist and why they are an improvement. Along with this it will also cover how to control what users are allowed to see, as well as explaining systems for creation of custom website forms, and finally it will cover the graphical displaying of statistics.
%conclusion




\subsection{Marking Systems}


\subsubsection{Lecturer Based}
The way lecturer based marking works is that students complete their assignment, the lecturer or tutor marks it and provides result in a timely manner with useful feedback which can be majorly important in helping students improve \cite{tang_investigating_2011}.
The advantage of this style of marking is that students can get useful feedback from their lecturer that can help improve their learning.  An article study \cite{higgins_conscientious_2002} found that of the students, when surveyed 82\% agreed to the question "I pay close attention to the comments I get" in response to assignment feedback.


A downside to this style of marking is that as the number of students increases on courses the amount of time required to mark assignments consequently takes longer and in some cases this can actually cause marked assessments to be scrapped completely as they take too long to get feedback to students \cite{brown_assessment_1999}.




\subsubsection{Peer Based}
To cope with increasing class sizes some courses are beginning to move towards peer marking. Peer marking system works by having students assess each other and in some cases  the students produced their own marking criteria \cite{orsmond_use_2000}.


This style allows  students to gain experience in evaluating other people's work, which some graduates feel is a necessary skill to possess. \cite{langan_insights_????}. Peer marking also deals with increasing amounts of students very well- this is because as the number students increase the number of markers also increases!


Peer marking however has its own set of problems - for example, "Students may have a less well developed sense of the criteria compared to the lecturer which could lead to a lack of reliability of student marking." \cite{orsmond_use_2000}.






\subsection{Digital Marking Systems}


\subsubsection{Reasons for Digital Marking}
Digital marking systems are designed to mirror the current paper based marking systems but with the advantage of the electronic environment \cite{heinrich_online_2003}. These systems help to reduce the increasing workload caused by more and more students taking courses. Along with this it also allows administrative tasks associated with coursework to be automated enabling more time for other tasks.\cite{joy_effective_1998}.


For students digital marking is great as it allows for quick feedback as the assessor is able provide students feedback immediately after they have written it up, instead of having to wait for a class to receive it. In one study\cite{dahl_turnitin_2007} they found that 78\% of students would like get their feedback  electronically .


According to the highlighted article \cite{derby_duplication_2008} plagiarism is on the rise amongst student. This is where digital marking can help to reduce plagiarism as the programme can  do what a human marker cannot. They can compare a submission thousands of documents and judge if a person has plagiarised. They can also help to show patterns in assignments and marks that normally might go unnoticed. 


\subsubsection{TurnItIn}
There currently exists an on-line electronic plagiarism system called TurnItIn, \cite{_turnitin_????} currently being used by many universities around the world. It allows students to upload their essay assignments online. It then checks for plagiarism in the document by searching the internet and using a large database of documents. After it processes the document it assigns a plagiarism percentage and highlights any areas that were plagiarised. Lecturers can then login and view all the submitted documents and mark them .\\
Current research highlighted \cite{dahl_turnitin_2007} conducted a questionnaire and found that students felt that the system was easy to use and more convenient than having to provide paper copies. It also found that 50\% of students strongly agreed and 33.3\% just agreed that they preferred to have their grade shown online rather than have a cover sheet.


\subsubsection{BOSS System}
The BOSS system  was developed at the University of Warwick to help deal with their problem of having too many students for the number of staff and yet wanting students to have accurate and quickly available feedback \cite{joy_boss_2005}.
It is an electronic submission and assessment system created to allow computing science students to submit their programming assignments and have them tested and marked online \cite{joy_effective_1998}. The system is not designed to remove human markers completely, instead simply "assist the instructor in achieving a quicker, more accurate and more consistent assessment of programming assignment"\cite{joy_boss_2005}.


When a file is first submitted it is run through a plagiarism check to make sure that the submission is actually the student’s own work. It also checks that the submission passes pre-set tests to make sure it works. After this it goes into the evaluations stage, since evaluation attributes of code can be very subjective what the second step does is generate metrics about the submitted program. Some of these metric are a number of comments and percentage of methods declared \cite{joy_effective_1998}, which will help human markers evaluate the submission quicker. 


\newpage
\subsection{User Access Views}


Controlling the view that users have, based on their access level, is common practice. Social media websites for instance allow users to limit what others  can see, through the use of a privacy setting. This means that if another user is a friend then they are allowed to see their whole feed, while other users may only see their profile name.


The patent highlighted \cite{baker_system_1997}, describes a system of limiting user web page access through the use of relation databases. The system would work by using two databases; one would hold a list of all the url's and associated access level, while the second database would hold all the user id's along with their assigned access level. When a user requests a webpage, the access level for that webpage and the user are looked up. If the users do not have the appropriate access they are denied permission to load the page and depending on implementation may be redirected to another webpage
The design of this system is well suited for scalability since no matter how large the two datasets are only one piece of data is need from each database to confirm whether a user is allowed access.




\subsection{Custom Input Forms}


\subsubsection{SurveyMonkey}
Survey Monkey \cite{finley_surveymonkey_1999} is an example of custom web forms being created by users. Founded by Ryan Finley in 1999 Survey Monkey enables users to create their own surveys and easily distribute them. It builds the surveys by letting the user select the contents of the question and what the response type will be: The user can also decide if the responses are completely anonymous by default and the participants ip address is stored when they complete the survey. The users can continue to add as many questions as they would like, even after the survey is initially created. After designing the survey the user chooses how they would like to have their survey distributed. The available options that can be selected are a web link, social media, email or embeddable on a website \cite{waclawski_how_2012} .


When participants complete the survey their results are immediately stored and the results of the survey are visible to the user by login into their account on surveymonkey. They can choose to look at the responses individually or look at metrics about how participants responded.




\subsubsection{Customizing Forms In Electronic Mail Systems}
\noindent
A patent \cite{holt_customizing_2006} describes a process for user-customisable forms in an e-mail system where the administrator selects custom field types and behaviours.  For example current e-mail forms have


\begin{wrapfigure}[7]{r}{0.5\textwidth}
\vspace*{-\baselineskip}
\begin{figure}[H]
 \includegraphics[width=0.7\textwidth]{images/emailform.png}
	\caption{Example Form (Patent \cite{holt_customizing_2006})}
	\label{fig:emailform}
\end{figure}
\end{wrapfigure}


\noindent
a field for address, subject and one for the the actual message to be sent. While an example of what the patent is suggesting can be seen in figure \ref{fig:emailform}, it shows the adding of additional fields allowing for wide variety of form to be created and not limit the  users to use the few forms that are precreated.\\
This increased flexibility in email forms would allow for easier  interpretation of messages, making responding or providing information via email a lot simpler and quicker.




\subsection{Development Tools}
\subsubsection{D3}
D3 \cite{bostock_d3.js_????} is a javascript library, which was designed for the creation of interactive visualisations of data and was first developed in 2011. D3 uses precreated Javascript functions to create scalable vector graphics (SVG's) which are embedded into the html of websites."SVG is a language for describing two-dimensional graphics in XML"\cite{ferraiolo_scalable_2000} and can have displays changed  using Cascading Style Sheet (CSS).  \\
Datasets can also be bound to an SVG allowing for a visual way to interpret the dataset, and as the dataset changes the SVG will be changed allowing for a dynamic display. 


\subsubsection{CakePHP}
CakePHP is an opensource framework for php, it is developed to help with rapid development of web application and make them simpler,fast and less complex to build \cite{_cakephp_????}.
  



%%%%%%%%%%%%%%%%%%%%%%%%%%%%%%%%%%%%%%%%%
%										%
%     		 Requirements				%
%										%
%%%%%%%%%%%%%%%%%%%%%%%%%%%%%%%%%%%%%%%%%


\newpage
\section{Requirements}
\newcounter{requirement} \stepcounter{requirement}


\subsection{Functional}
Requirements for the system are each given an idea depending on the type of requirement: FR for functional requirements, NFR for non-functional requirement and SR for system requirements.\\
Along with this, each requirement has a description stating what the requirement is and a priority. The priority value can be low, medium or high, which shows which requirements will be implemented first into the system.\\
For this project I will be attempting to implementing all of the high priority functional and non-functional requirements, I will also try and implement as many medium and low priority requirements that I can starting with ones that best improve the system. 


\def\arraystretch{1.5}
\subsubsection{User Requirements}
Functional requirements also include an access column which defines what users should be able to use. Some items are restricted to lecturers as some requirements should only be be usable by lecturers and lab-helpers and not by students. The table is sorted first by access level starting with widest allowed access then sorted in access order 1 - 4, it is secondly sorted by priority.

The access levels are: 1-Admin, 2-Lecturers, 3-Lab Helpers and 4-Students


\begin{spacing}{1.5}
\begin{longtable}{|p{0.09\linewidth}|p{0.6\linewidth}|p{0.1\linewidth}|
p{0.1\linewidth}|}
\caption{Functional User Requirements} \label{table:funct-user} \\
\hline


\textbf{ID} & \textbf{Requirement} & \textbf{Access} & \textbf{Priority}\\
\hline \hline


\rid{FR} & Should have to login to view system & 1,2,3,4 & High\\ \hline
\rid{FR} & Should have accounts created for them & 1,2,3,4 & High\\ \hline
\rid{FR} & Should be able to change password & 1,2,3,4 & High\\ \hline
\rid{FR} & Should be able to login using university ID & 1,2,3,4 & Low\\ \hline
\rid{FR} & Should be able to logout & 1,2,3,4 & High \\ \hline

\rid{FR} & Should be able to remove students from courses & 1, 2 & High\\ \hline
\rid{FR} & Should be able to update student accounts & 1,3 & Low \\ \hline

\rid{FR} & Should be able to look up students in lab & 2,3 & High\\ \hline
\rid{FR} & Should be able  to select students from lab list & 2,3 & High\\ \hline
\rid{FR} & Should be able to leave comments about students & 2,3 & High\\ \hline
\rid{SR} & Should be able to save marks & 2,3 & High\\ \hline
\rid{SR} & Should be able to update marks & 2,3 & High\\ \hline
\rid{SR} & Should be able to delete marks & 2,3 & High\\ \hline
\rid{FR} & Should be able to search for student by name & 2,3 & Medium\\ \hline
\rid{FR} & Should be able to mark student even if they are not in the system & 2,3 & Medium \\ \hline

\rid{FR} & Should be able to assign students to courses & 1 & Medium\\ \hline
\rid{FR} & Should be able to assign lectures to courses & 1 & Medium \\ \hline

\rid{FR} & Should be able to create marking schemes & 2 & High\\ \hline
\rid{FR} & Should display generated stats & 2 & High\\ \hline
\rid{FR} & Should be able to see submited marks & 2 & High\\ \hline
\rid{FR} & Should be able to generate end of year spread sheets & 2 & Medium\\ \hline
\rid{FR} & Should allow editing of students in class & 2 & Medium\\ \hline
\rid{FR} & Should be able to create peer marking scheme & 2 & Medium\\ \hline
\rid{FR} & Should be able to look at students stats & 2 & Medium\\ \hline
\rid{FR} & Should be able to set what parts of the marking scheme students can see & 2 & Medium\\ \hline
\rid{FR} & Should be able to update marking scheme & 2 & Medium \\ \hline
\rid{FR} & Should be able to delete marking schemes & 2 & Medium\\ \hline
\rid{FR} & Should be able to able to assign students to set labs & 2 & Low \\ \hline
\rid{FR} & Should be able to set penalties for late marking & 2 & Low \\ \hline
\rid{FR} & Should able to export to vision & 2 & Low\\ \hline

\rid{FR} & Should be able to access Marking Scheme & 3 & High\\ \hline
\rid{FR} & Should be able to enter selected students mark & 3 & High\\ \hline
\rid{FR} & Should be able to submit student mark & 3 & High\\ \hline
\rid{FR} & Should be able to select the lab they are helping in & 3 & High\\ \hline

\rid{FR} & Should be able to see current mark & 4 & High\\ \hline

\rid{FR} & Should show different displays depending on access level & & High\\ \hline
\rid{FR} & Should load students current lab mark scheme & & High\\ \hline
\rid{FR} & Should apply penalty for late lab completion & & High\\ \hline
\rid{FR} & Should create a set of useful stats based on lab & & High\\ \hline
\rid{FR} & Should store what class student belong too & & High\\ \hline
\rid{FR} & Should have a list of all students in class & & High\\ \hline

\end{longtable}
\end{spacing}
\setcounter{requirement}{1}


\newpage
\subsection{Non-Functional Requirements}

Table \ref{table:non-func} lists all the non-functional requirements for the development of the system, they are ranked in order of priority.

\begin{spacing}{1.5}
\begin{longtable}{|p{0.1\linewidth}|p{0.7\linewidth}|p{0.1\linewidth}|}
\caption{Non-Function Requirements} \label{table:non-func}\\
\hline
\textbf{ID} & \textbf{Requirement} & \textbf{Priority}\\
\hline \hline

\rid{NFR} & Should have all person data encrypted & High\\ \hline
\rid{NFR} & Should update stats as marks are entered & High\\ \hline
\rid{NFR} & Should take less than 2 seconds to generate stats  & High\\ \hline
\rid{NFR} & PHP Should use prepared statements & High\\ \hline
\rid{NFR} & Should be dynamically designed & High\\ \hline
\rid{NFR} & HTML, CSS and Javascript should be validated & High\\ \hline
\rid{NFR} & Should make sure inputs are valid & High\\ \hline
\rid{NFR} & Should prevent SQL Injection & High\\ \hline

\rid{NFR} & Should function on a wide variety of smartphones and tablets & Medium\\ \hline
\rid{NFR} & Should be able to handle a large number of users without any faults & Medium\\ \hline
\rid{NFR} & Should make sure passwords contain alphanumerics and have a minimum and maximum length  & Medium\\ \hline
\rid{NFR} & Should autosave marks as they are entered & Medium\\ \hline
\rid{NFR} & Should record what lab help marked what student & Medium\\ \hline
\rid{NFR} & Should list all students that did not attend the lab & Medium\\ \hline
\rid{NFR} & Should track how long it takes to mark a student & Medium \\ \hline

\rid{NFR} & Should have disability options (Increase text size, colour layout) & Low\\ \hline
\rid{NFR} & Should be readable by screen readers & Low\\ \hline
\rid{NFR} & Should take less than 2 second to load student marking scheme & Low\\ \hline
\rid{NFR} & Should be able to group marked people & Low \\ \hline
\rid{NFR} & Should retrieve student images from university system & Low\\ \hline
\rid{NFR} & Should backup database regularly & Low\\ \hline


\end{longtable}
\end{spacing}


\setcounter{requirement}{1}




%%%%%%%%%%%%%%%%%%%%%%%%%%%%%%%%%%%%%%%%%
%										%
%     		Testing & Evaluation		%
%										%
%%%%%%%%%%%%%%%%%%%%%%%%%%%%%%%%%%%%%%%%%
\newpage


\section{Strategy for testing and evaluation}


\subsection{Testing}
During each sprint unit tests will be created and run on modules of code to make sure that they function correctly, and can deal with a wide variaty of unexpected or extreme cases. It will help test if the function is robust enough  and the system is ready for the next module to be developed.

Each sprint will have set requirements that are to be implemented by the end of the sprint. These requirement will each have a test case that will be run at the end of the sprint to make sure that it is successfully implemented.


\subsection{Evaluating}
To evaluate properly how successful I have been at developing a new Lab Marking System I will conduct a usability case study. Lecturers, lab helpers and students will be asked to use the systems and provide feedback, to help evaluate the system and discover what improvements can be made.

To evaluate how effective the code is I will create test cases. These will test how efficient the code is at running functions and help find areas for future improvement in the system.

To evaulate the whole system I will use the objectives I defines in section \ref{section:object}, each of the objectives is testable and testing them will help evalute successful I was at implementing the completed system and what will need future improvement.





%%%%%%%%%%%%%%%%%%%%%%%%%%%%%%%%%%%%%%%%%
%										%
%     		Project Plan				%
%										%
%%%%%%%%%%%%%%%%%%%%%%%%%%%%%%%%%%%%%%%%%
\newpage
\section{Project Plan and Risk Analysis}


\subsection{Project Plan}


The Gantt chart and its accompanied task table for this dissertation can be seen in appendix (\ref{appendix:gantt}), I used gantt project to create it unfortunatly it does not allow tasks to stop at weekends leaving gaps between tasks which should not be there. It is broken down into 5 stages: Design, Development, Evaluation, Dissertation and Poster. Each of these stages has multiple tasks that are expected to be completed during the course of the stage. A summary of each stage and their key tasks are stated below.\bigskip

\noindent
\textbf{Design Stage} Starts at the end of semester 1 to allow myself time to complete other course work. In this stage I will create mock-ups for the user interface, a database schema and Unified Modeling Language (UML) diagram, to help improve the development time of the system and create an idea of what the system will look like. Also along with this I will also be deciding requirements will be added in each of my sprints in the next stage.\bigskip

\noindent
\textbf{Development Stage:} Starts once the holidays are over. It is consists of five sprints, four of which are one week long and one that is two weeks long, each sprint has its own aim. 

\noindent Sprint 1 - will be aimed toward developing the basic user interface that will be used for the other sprints.\\ 
Sprint 2 - Is for develop the back-end of the system, developing the database login system and any other functionality that will be needed for the next sprint.\\
Sprint 3 - This is the two week sprint this is because the aim is to enable the creation of custom marking schemes and I do not know what exact functionality will be required to make it work so I gave myself more time to develop it.\\
Sprint 4 - Aims to build on the previous sprint by developing lab helpers functionality.\\ 
Sprint 5 - Is for general improvements of the system mostly to focus on the user interface.\\
After sprints 2 and 5 there is a planned week for evaluation and documentation for the functionality implemented in the sprints that have occurred.\bigskip

\newpage
\noindent
\textbf{Evaluation Stage: } This stage occurs at the final two weeks before the draft handin, I will consist of the creation of a usability case study (UCS) using the final result of the development stage, The UCS will be run and using the evaluated results to develop improvements for the system in the future. During the course of this stage I will also be completing the draft of my dissertation ready for the handin at the start of the next stage\bigskip

\noindent
\textbf{Final Deliverable Stage:} This stage start with the handin of my draft dissertation, it is followed by a weeks break to allow for my supervisor to assess my draft. After that using the feedback I receive from both my supervisor and my second reader I will go about improving my dissertation so that it is of a high enough standard for the final handin.\bigskip

\noindent
\textbf{Poster Stage:} This stage will be entirely dedicated to the design and creation of my dissertation poster.






%%%%%%%%%%%%%%%%%%%%%%%%%%%%%%%%%%%%%%%%%
%										%
%     		Risk Analysis				%
%										%
%%%%%%%%%%%%%%%%%%%%%%%%%%%%%%%%%%%%%%%%%
\newpage
\subsection{Risk Analysis}

The risks relating to this dissertation are shown in table(\ref{table:risks}), each risk has an associated Likelihood, Impact and a Mitigation Strategy. Likelihood values are low, moderate and high, Impact values can be low, moderate or severe. The table is sorted by likelihood and then by the impact.

\begin{spacing}{1.5}
\begin{longtable}{|p{0.05\textwidth}|p{0.25\textwidth}|p{0.15\textwidth}|p{0.1\textwidth}|p{0.33\textwidth}|} 
\caption{Risk Analysis} \label{table:risks} \\
\hline

\textbf{ID} & \textbf{Risk} & \textbf{Likelihood} & \textbf{Impact } & \textbf{Mitigation Strategy}\\
\hline

\rid{R} & Browsers compatibility & High & Servere & Constantly test browser compatibility after each sprint\\ \hline

\rid{R} & Other university coursework deadlines & High & Moderate & Make sure to allocate adequate amount of time to other courses.\\ \hline



\rid{R} & Lecturers can’t create custom marking schemes & Moderate & Severe & Create selectable forms that lecturers can edit.\\ \hline

\rid{R} & Requirements changed & Moderate & Moderate & Evaluate requirements before starting development phase and evaluate requirements regularly during project to notice any required changes before it causes a major issue \\ \hline

\rid{R} & System speed is slow & Moderate & Moderate & Make sure system is well factored and follow coding standards \\ \hline

\rid{R} & Users cannot understand the system & Moderate & Moderate & Usability Study is being run at the end of the project to evaluation future improvements\\ \hline

\rid{R} & Used libary get updated and causes error & Moderate & Moderate & Read update information before upgrading to understand where errors might occur\\ \hline

\rid{R} & Admins can’t assign courses & Moderate & Moderate & Assign courses as part of test data and make note as a future improvement to the system  \\ \hline

\rid{R} & Superviser unable to make meeting & Moderate & Low & Contact supervisor and arrange another meeting time \\ \hline

\rid{R} & Cannot get D3 to work & Moderate & Low & Just display stats as plain text\\ \hline



\rid{R} & Personal Injury & Low & severe & Development would have to be delayed and discussions made with supervisor about how to continue\\ \hline

\rid{R} & Loss of data & Low & severe & Back-ups will be stored throughout the project\\
\hline

\rid{R} & Family Issue & Low & severe  & Inform MACS and discuss what options are available depending on the situation \\ \hline

\rid{R} & Users can't login & Low & severe & Major testing will be done of the login functionality including multiple unit tests\\ \hline

\rid{R} & Users can't log out & Low & Moderate & Login will be using sessions which will timeout after a set time so if the user cannot log out then the system do it for them\\ \hline

\rid{R} & Server hosting the project crashes & Low & Moderate & Contact the host and enquire about what has happened\\ \hline

\rid{R} & Superviser Leaves & Low & Moderate & Inform alisdair and request a new supervisor\\ \hline

\rid{R} & Software Licences Expire & Low & Low & Check licences for all software i will be using during the project to make sure they are valid for length of project\\ \hline

\rid{R} & Issues with transport to university  & Low & Low & Use SSH to access any required information at the university  \\ \hline

\end{longtable}
\end{spacing}






%%%%%%%%%%%%%%%%%%%%%%%%%%%%%%%%%%%%%%%%%
%										%
%     			 P.L.E.S				%
%										%
%%%%%%%%%%%%%%%%%%%%%%%%%%%%%%%%%%%%%%%%%
\newpage


\section{P.L.E.S Issues}


\subsection{Professional Issues}
The professional part of this project will be done by following coding standards for the languages that I decide to use.

As this project will be a web application I will use external website to make sure that both the html and css in the system are both fully validated.


I will do my best to make sure that any personal information stored about people in the system will secure and encyption, to make sure users feel safe in using the system. 

The project will be provided with a  user guild for the system so that people unfirmiliar with it will be able to pick it up and use it quickly. I will also create  a develper guild, that will detail how the system is designed, what structures are used and describe what method is the system do. This is to help future developement of the system by other people, as it gives them an 


\subsection{Legal Issues}
There are multiple legal issues relating to this project. The most important one is the Data Protection Act. Since the systems will be designed to store data about students I will have to make sure that all data is encrypted and securely stored.

I will make sure that I follow the terms and services for any software or libraries I used as part of the project, to ensure I am using it in a legal fashion.


The system will be made open source, this is to allow other people to look at and improve the system once I have completed it. 


\subsection{Ethical Issues}
A major ethical requirement of this project is to do with the storage of students personal information on a digital system; to deal with this issue I should consult the data protection act.

Another issue that is raised by this project is making sure that students are not deceived and that the marks they see are actually the ones they have received.

The largest ethical issue I will have to deal with is my usability case study, which will be after the development stage. All particapants will be required to fill out a consent form before they take part, to make sure they know what they are taking part in. I will also have to make sure that I am impartial and that none of the questions in the case study are leading, so that useful data can be recieved from the study to help future development.

The last ethical issue relating to this project is privacy. I have to  ensure that students, when they are online, are able to see only their own mark and cannot see another student's mark. It should also be made such that final grades are only visible to the lecture and student keeping it confidential from lab helpers.

%Case Study
%Security breaches & data contained
%secure data
%bugs and other parts


\subsection{Social Issues}
A few social issues are raised by this project. Such as if students can see the mark they have received straight away, will lab helpers feel pressurised into giving higher grades.\\
Will this system result in a reduction of lab helpers being required to mark labs? If the system speeds up the time to mark students work, less lab helpers may be required to run labs, resulting in people looking for work.






%%%%%%%%%%%%%%%%%%%%%%%%%%%%%%%%%%%%%%%%%
%										%
%     		Bibliography				%
%										%
%%%%%%%%%%%%%%%%%%%%%%%%%%%%%%%%%%%%%%%%%

\newpage
\printbibliography[heading=bibintoc]
\let\cleardoublepage\clearpage



%%%%%%%%%%%%%%%%%%%%%%%%%%%%%%%%%%%%%%%%%
%										%
%     		Appendice					%
%										%
%%%%%%%%%%%%%%%%%%%%%%%%%%%%%%%%%%%%%%%%%

\begin{appendices}

\let\cleardoublepage\clearpage
\section{Project Plan}
\label{appendix:gantt}
\subsection{Gantt Chart}
\centering {\includegraphics[height=0.9\textheight]{images/projectplan.jpg}}

\includepdf[scale=0.85, pages={2}, pagecommand=\subsection{Task List}]{pdf/projectplan.pdf}

\end{appendices}



\end{document}