\documentclass[12pt]{article}  % [12pt] option for the benefit of aging markers
\usepackage{amssymb,amsthm}    % amssymb package contains more mathematical symbols
\usepackage{graphicx}          % graphicx package enables you to paste in graphics

%%%%%%%%%%%%%%%%%%%%%%%%%%%%%%%%%
%
%    Page size commands.  Don't worry about these
%
\setlength{\textheight}{220mm}
\setlength{\topmargin}{-10mm}
\setlength{\textwidth}{150mm}
\setlength{\oddsidemargin}{0mm}

%%%%%%%%%%%%%%%%%%%%%%%%%%%%%%%%%%%%%%%%%%%%%%%%%%%%%%%%%%%%%%%
%
%    Definitions of environments for theorems etc.
%
\newtheorem{theorem}{Theorem}[section]          % Theorems numbered within sections - eg Theorem 2.1 in Section 2.
\newtheorem{corollary}[theorem]{Corollary}      % Corollaries etc. will be counted as Theorems for numbering
\newtheorem{lemma}[theorem]{Lemma}              % eg Lemma 3.1, ... Theorem 3.2, ... Corollary 3.3.
\newtheorem{proposition}[theorem]{Proposition}
\newtheorem{conjecture}[theorem]{Conjecture}

\theoremstyle{definition}
\newtheorem{definition}[theorem]{Definition}

\theoremstyle{remark}
\newtheorem{remark}[theorem]{Remark}
\newtheorem{example}[theorem]{Example} 

%%%%%%%%%%%%%%%%%%%%%%%%%%%%%%%%%%%%%%%%%%%%%%%
%
%        Preamble material specific to your essay
%
\title{Lab Marking system}
\author{Lewis McNeill\\
F1.0GP2 Project\\
supervised by
Peter J King}

\begin{document}
\maketitle

\newpage                     % optional page break
\begin{abstract}

Write a short abstract of your essay here.



\end{abstract}

\newpage                     % optional page break
\tableofcontents

\newpage                     % optional page break
\section{Introduction}\label{s:intro}
%
% The \label command is optional, but useful.  To cross-refer to a section/theorem/equation etc.
% labelled by \label{key}, use \ref{key}.  For example: Equation (\ref{eq:key}) follows from Theorem \ref{th:key}.

\subsection{Background}\label{ss:back}

\begin{theorem}\label{th:important}
 An important mathematical theorem
\end{theorem}

See \cite[Chapter 3]{Cassels} or \cite{Bovey} for more details.

\begin{proof}
The `proof' environment provides a standard beginning and end for your proofs,
and takes care of the spacing around them.
\end{proof}

\section{Graphics}

Graphics files in PDF or JPEG format can be pasted in using the code
below (remove the \%\ signs and edit according to your needs).

%  \begin{center}
%  \includegraphics{myfile.pdf}
%  \includegraphics{myfile.jpg}
%
%  \includegraphics[width=175mm]{myfile.pdf}    if you need to change the size
%
%  \includegraphics[width=150mm, height=100mm]{myfile.pdf}  if you need to change the size in both directions
%
%  \includegraphics[angle=90]{myfile.pdf}     if you need to rotate the image
%
%  \includegraphics[angle=270,width=120mm]{myfile.pdf}  if you need to rotate and change size
%
%  \end{center}

\section{Lists and Tables}

It is often useful to display information in the form of a list or a table.
Here will display some examples of:

\begin{itemize}
\item A function $f$;
\item Its derivative $f'$; and
\item Its indefinite integral $\int f$.
\end{itemize}

\begin{center}              % center the table
\begin{tabular}{|r|r|r|}    % three columns, each right-justified
\hline                      % horizontal line between rows
$f$ & $f'$ & $\int f$ \\    % header row
\hline
$x^3$ & $3x^2$ & $x^4/4$ \\
$\cos(x)$ & $-\sin(x)$ & $\sin(x)$ \\
$e^x$ & $e^x$ & $e^x$\\
\hline
\end{tabular}               % close the table
\end{center}                % exit the center environment

\section{Conclusions}\label{s:conc}



%%%%%%%%%%%%%%%%%%%%%%%%%%%%%%%%%%%%%%%%%
%
%     Bibliography
%
%     Use an easy-to-remember tag for each entry - eg \bibitem{How97} for an article/book by Howie in 1997
%     To cite this publication in your text, write \cite{How97}.  To include more details such as
%     page, Chapter, Theorem numbers, use the form \cite[Theorem 6.3, page 42]{How97}.
%
\begin{thebibliography}{99}

% 
% The usual convention for mathematical bibliographies is to list alphabetically
% by first-named author (then second, third  etc. author then date)
% websites with no author names should go by the site name
%



% Typical layout for reference to a journal article
%
\bibitem{Bovey}
J. D. Bovey, M. M. Dodson,                         % author(s)
The Hausdorff dimension of systems of linear forms % article name
{\em Acta Arithmetica}                             % journal name - italics
{\bf 45}                                           % volume number - bold
(1986), 337--358.                                   % (year), page range

% Typical layout for reference to a book
%
\bibitem{Cassels}
J. W. S. Cassels,                                  % author(s)
{\em An Introduction to Diophantine Approximation},% title - italics
Cambridge University Press, Cambridge, 1965.       % Publisher, place, date.

% Typical layout for reference to a website
%
\bibitem{GAP}
The GAP Group, GAP -- Groups, Algorithms, and Programming,  % Site name
Version 4.5.6; 2012. % other information
(http://www.gap-system.org)  % URL


% Typical layout for reference to an online article
%
\bibitem{Howie}
J. Howie,                                            % author(s)
{\em Generalised triangle groups of type $(3,5,2)$}, % article name - italics
http://arxiv.org/abs/1102.2073                       % URL
(2011).                                              % (year)
\end{thebibliography}
\end{document}
