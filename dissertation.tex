\documentclass[12pt]{article}  % [12pt] option for the benefit of aging markers
\usepackage{amssymb,amsthm}    % amssymb package contains more mathematical symbols
\usepackage{graphicx}          % graphicx package enables you to paste in graphics

\usepackage{tabularx}
\usepackage{setspace}

%%%%%%%%%%%%%%%%%%%%%%%%%%%%%%%%%
%
%    Page size commands.  Don't worry about these
%
\setlength{\textheight}{220mm}
\setlength{\topmargin}{-10mm}
\setlength{\textwidth}{150mm}
\setlength{\oddsidemargin}{0mm}

%%%%%%%%%%%%%%%%%%%%%%%%%%%%%%%%%%%%%%%%%%%%%%%%%%%%%%%%%%%%%%%
%
%    Definitions of environments for theorems etc.
%
\newtheorem{theorem}{Theorem}[section]          % Theorems numbered within sections - eg Theorem 2.1 in Section 2.
\newtheorem{corollary}[theorem]{Corollary}      % Corollaries etc. will be counted as Theorems for numbering
\newtheorem{lemma}[theorem]{Lemma}              % eg Lemma 3.1, ... Theorem 3.2, ... Corollary 3.3.
\newtheorem{proposition}[theorem]{Proposition}
\newtheorem{conjecture}[theorem]{Conjecture}

\theoremstyle{definition}
\newtheorem{definition}[theorem]{Definition}

\theoremstyle{remark}
\newtheorem{remark}[theorem]{Remark}
\newtheorem{example}[theorem]{Example} 

%%%%%%%%%%%%%%%%%%%%%%%%%%%%%%%%%%%%%%%%%%%%%%%
%
%        Preamble material specific to your essay
%
\title{Lab Marking system \\~\\  \large{Heriot-Watt University} \\~\\ Final Year Dissertation}
\author{Lewis McNeill\\
supervised by
Peter J King}

\begin{document}
\maketitle

\newpage 

\textbf{\Large{Declaration}} \\[2em]
I, Lewis Francis McNeill, confirm that this work submitted for assessment is my own and is expressed in my own words. Any uses made within it of the works of other authors in any for (e.g., ideas, equations, figures, text, tables, programs) are properly acknowledged at any point of their use. A list of the references employed is included.
\\
\\
Signed: Lewis McNeill
\\
Date: \today

\newpage                     % optional page break
\begin{abstract}

The project aim is to develop a web application that will be used to improve marking of computing labs. The application will be designed to be used by Students to quickly know their grade, by Lab Helpers to easily mark labs and Lecturers to see marking immedatly as it is done.



\end{abstract}

\newpage                     % optional page break
\tableofcontents




\newpage                     % optional page break
\section{Aims, Objectives and Project Description}

\subsection{Aim}

\subsection{Objectives}

\newpage
\section{Literature Review}

\newpage
\section{Requirements}
\subsection{System Requirements}


\begin{tabularx}{\textwidth}{|l|X|X|X|X|}
\hline
  \textbf{ID} & \textbf{Requirement} & \textbf{Type} & \textbf{Description} & \textbf{Priority} 
\\
\hline
R1&Test&Test&Test&Test\\ \hline
R2&Test&Test&Test&Test\\ \hline


\end{tabularx}

\subsection{Usability Requirements}


\newpage
\section{Strategy for testing and evaluation}

\subsection{Testing}
\doublespacing
Testing and evalutation of the system will be done in two parts. To start with throughout the development of the the system unit tests will be used to make sure that the system is robust and functional. \\

\subsection{Evaluating}
Once the system is completed a useability case study will be conducted to evaluate how successful the development of the marking system  was and how later versions can be improved.

\singlespacing

\newpage
\section{Project Plan and Professional, Legal, Ethical and Social Issues}








%%%%%%%%%%%%%%%%%%%%%%%%%%%%%%%%%%%%%%%%%
%
%     Bibliography
%
%     Use an easy-to-remember tag for each entry - eg \bibitem{How97} for an article/book by Howie in 1997
%     To cite this publication in your text, write \cite{How97}.  To include more details such as
%     page, Chapter, Theorem numbers, use the form \cite[Theorem 6.3, page 42]{How97}.
%
\newpage
\begin{thebibliography}{99}

% 
% The usual convention for mathematical bibliographies is to list alphabetically
% by first-named author (then second, third  etc. author then date)
% websites with no author names should go by the site name
%



% Typical layout for reference to a journal article
%

\bibitem{EssayType}
E. Heinrich, Y. Wang,
{\em Online Marking of Essay-type Assignments};
2003.(http://www-ist.massey.ac.nz/MarkTool/Publications/EdMedia2003Onscreen.pdf)

\bibitem{Bovey}
J. D. Bovey, M. M. Dodson,                         % author(s)
The Hausdorff dimension of systems of linear forms % article name
{\em Acta Arithmetica}                             % journal name - italics
{\bf 45}                                           % volume number - bold
(1986), 337--358.                                   % (year), page range

% Typical layout for reference to a book
%
\bibitem{Cassels}
J. W. S. Cassels,                                  % author(s)
{\em An Introduction to Diophantine Approximation},% title - italics
Cambridge University Press, Cambridge, 1965.       % Publisher, place, date.

% Typical layout for reference to a website
%
\bibitem{GAP}
The GAP Group, GAP -- Groups, Algorithms, and Programming,  % Site name
Version 4.5.6; 2012. % other information
(http://www.gap-system.org)  % URL


% Typical layout for reference to an online article
%
\bibitem{Howie}
J. Howie,                                            % author(s)
{\em Generalised triangle groups of type $(3,5,2)$}, % article name - italics
http://arxiv.org/abs/1102.2073                       % URL
(2011).                                              % (year)
\end{thebibliography}
\end{document}


